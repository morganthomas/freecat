\documentclass{article}

\title{FreeCat design document}
\author{Morgan Thomas}

\begin{document}

\maketitle

FreeCat is a programming language, a system of logic and computer algebra, and an approach to type theory and the foundations of math. It is based on the hypothesis that we can create simpler and more general approaches to problems in these areas when we do not try to rule out all possibility of logical inconsistency.

FreeCat is an open source, non-profit project whose primary aim is to give people a new way to develop software. The hope is that FreeCat will help people reduce their costs for developing software, while achieving higher quality.

A secondary aim is to describe a new approach to understanding the foundations of math. The hope is that people will find this approach simpler, conceptually clearer, and capable of expressing more concepts, compared to alternatives.

FreeCat doesn't aim to be the best approach to solving everybody's problem. For now the focus is on making it a good approach to solving some problems.

The conception and justification of FreeCat is rooted in a holistic, interdisciplinary approach to analysis. This holistic methodology assumes that there are deep connections between fundamental problems in different technical disciplines. It assumes in essence that we can clarify our thinking about technical problems in general by developing theories which can systematize solutions to technical problems in general.

This design document aims to be a self-contained conceptual explanation of FreeCat. Prior experience in any of the following areas is helpful, but not required, for understanding this document: functional programming, type theory, holistic analysis, foundations of math, mathematical logic, philosophical logic, category theory, abstract/universal algebra, programming language theory.

\end{document}
